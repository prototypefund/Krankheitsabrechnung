% au.tex
% Tex-Datei zum Drucken der Arbeitsunf�higkeitsbescheinigung auf
% dem gelb/roten Kassenformular
% Author Christoph Becker, Dauner Str. 21, 53539 Kelberg, Germany
% email: cgbecker@gmx.de
% 4. Febr. 2005 
% Prinziep:
% Das gesamte Document besteht nur aus EINEM Bild bzw. 'picture'
% Als Ursprungskoordinate wird die rechte untere Ecke 
% des Rahmens mit den Versichertenkartenleistungen verwendet.
% Die Ursprungskoordinate kann nicht an einem Papierrand liegen, weil
% LATEX einen Seitenrand von 0 ignoriert. Von daher ist es besser bei
% Formularen einen leicht erkennbaren Ursprungspunkt im Formular zu
% benutzen. Der Ursprungspunkt wird als Schnittpunkt von oberem und
% linkem Rand eingestellt. Die Bildgroesse kann (0,0) sein.
% Alle Felder des Formulars werden als Bildelemente (vom Typ Text) plaziert 
% (Siehe Helmut Kopka, Latex - Eine Einfuehrung, Bd 1, Kap. 6, Bilder
% Alle Felder des Forumlars werden im Bezug auf die Ursprungskoordinate
% des Bildes plaziert.
% Der KZV- bzw. KV-Stempel wird mit einem Offset eingestellt, hier:  
% \newcount\stempeloffsetx \stempeloffsetx=20
% \newcount\stempeloffsety \stempeloffsety=-10
% Dadurck kann der 'Stempel' durch einfaches Kopieren auch in andere
% Kassenformulare uebernommen werden (AU, HKP usw.). Es muss dann lediglich
% der Offset geaendert werden.
%
% Das Formular wurde auf einem EPSON 630 LQ-Nadeldrucker unter WindowsXP 
% mit TeXnicCenter (auf der Miktex-Cd, sehr zu empfehlen) entwickelt.
% wenn sich auf einem anderen Drucker oder unter Linux zeigt, dass die 
% Ursprungskoordinate nicht stimmt, so kann diese ueber   
% \oddsidemargin und \topmargin angepasst werden.
% Das Einsetzen der realen Daten erfolgt z.B. ueber ein Pythonskript
% welches die *Schluesselwoerter* sucht und gegen die jeweiligen Daten austauscht.

% EMACS-coding: iso-latin-9-with-esc
\documentclass[10pt,a5paper]{letter}
% a6paper gibt es nicht
\pagestyle{empty}
\oddsidemargin61mm
\topmargin28mm
\headheight0mm
\headsep0mm
\topskip0mm
\textwidth146mm
\textheight104mm
\footskip0mm
\paperwidth148mm
\paperheight105mm

\newcount\rechterrand 
\newcount\stempeloffsetx
\newcount\stempeloffsety


\begin{document}
\setlength{\unitlength}{1mm}
\begin{picture}(0,0)%(80,27) % x-Koordinate tut nichts zur Sache
% y-Koordinate schiebt d. 'Bild' nach unten
%\put(0,0){.+} %Ursprungskoordinate (Unterkante KVK-DatenBlock 
\rechterrand=-77 %beliebig dicht am re Papierrand! 
\stempeloffsetx=17
\stempeloffsety=-10
% KVK-Daten und ZA-Stempel einfuegen
% input-file kvkdaten.tex
% dient zur Darstellung des KVK-Datenblocks auf
% Kassenformularen
\put(\rechterrand,38){*KVKKrankenkasse*}
\put(\rechterrand,30){*KVKNameVorname*}
\put(-20,25){*KVKGebDatum*}
\put(\rechterrand,25){*KVKStrasse*}
\put(\rechterrand,20){*KVKPLZ-Ort*}
\put(\rechterrand,10){*KVKKassenNr*}
\put(-56,10){*KVKVersichertenNr*}
\put(-22,10){*KVKStatus*}
\put(\rechterrand,2){*KVKZa-KZVNr*}
\put(-49,2){*KVKVKgueltigbis*}
\put(-24,2){*KVKLeseDatum*}
%%%%%Ende KVK-Feldblock %%%%%%%%%%%%%%%%%%%%%%%%%%%%

%%%%%Begin KZV-Stempel %%%%%%%%%%%%%%%%%%%%%%%%%%%%%%
% KZV-Stempel setzen
% da dieser in verschiedenen Forumularen an verschiedenen Stellen
% gesetzt wird, wird hier mit einem Offset gearbeitet:
% -> nur der Offset ist anzupassen
\put(\stempeloffsetx,\stempeloffsety){\Large*KZVNr*}
\advance\stempeloffsetx by 22 \advance\stempeloffsety by 3
\put(\stempeloffsetx,\stempeloffsety){\scriptsize KZV}
\advance\stempeloffsety by -3
\put(\stempeloffsetx,\stempeloffsety){\small*Kobl-Tr*}
\advance\stempeloffsetx by -22 \advance\stempeloffsety by -5
\put(\stempeloffsetx,\stempeloffsety){\normalsize*Zahnarzt*}
\advance\stempeloffsety by -5
\put(\stempeloffsetx,\stempeloffsety){\normalsize*Zahnarztname*}
\advance\stempeloffsety by -5
\put(\stempeloffsetx,\stempeloffsety){\normalsize*ZahnarztStrasse*}
\advance\stempeloffsety by -5
\put(\stempeloffsetx,\stempeloffsety){\normalsize*ZahnarztPlZOrt*}
%%%%%%Ende KZV-Stempel %%%%%%%%%%%%%%%%%%%%%%%%%%%%%%%%%%
\put(\rechterrand,-11){*Erstbescheinigung*}
\put(-34,-11){*Folgebescheinigung*}

\put(\rechterrand,-20){*Arbeitsunfall*}
\put(-34,-20){*Durchgangsarzt*}

\put(-23,-29){*au-seit*}
\put(-23,-37){*au-bis*}
\put(-23,-45){*festgestelltam*}

\put(8,-62){*sonstUnfall*}
\put(8,-70){*Versorgungsleiden*}

% Diagnosefeld
\put(\rechterrand,-69){\parbox[t][50mm]{80mm}{*Diagnose*}}
% Erforderliche Massnahmen
\put(\rechterrand,-119){\parbox[t][20mm]{80mm}{*erforderlMassnahmen*}}
\end{picture}
\end{document}

%%% Local Variables:
%%% mode: latex
%%% TeX-master: rezept.tex
%%% End:

